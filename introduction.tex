\section{Introduction}

Ce rapport regroupe les informations concernant mon stage industriel effectué
entre ma seconde et troisième année d'école d'ingénieur, au sein de l'entreprise
Thales Corporate Services (abbrégée CST, maintenant Thales Global Services), une
filiale du groupe Thales.

Postuler pour ce stage est venu de mon désir de travailler au sein d'un grand 
groupe, en espérant faire partie d'une équipe travaillant sur un produit utilisé
par un relatif grand nombre d'utilisateur. Thales Corporate Services développe 
des logiciels destinés à être utilisé par l'ensemble du groupe Thales, ce qui
représente effectivement un large ensemble d'utilisateurs.

Ce stage à orientation technique a eu pour objectif de poursuivre les travaux 
d'un précédent stagiaire sur une plate-forme de test pour un des produits de la
firme. Le dit produit ainsi que le processus qui permet de le tester est 
détaillé dans la partie \ref{ThalesControl}. La plate-forme de test elle-même 
est donc essentiellement un outils d'aide au développement pour l'équipe 
derrière le produit final, avec laquelle j'ai travaillé en étroite 
collaboration.

Dans ce rapport, je vais d'abord présenter l'entreprise Thales Corporate 
Services, son rôle au sein de Thales ainsi les enjeux des projets sur lesquelles
elle travaille. Je poursuivrais ensuite avec ma mission technique, ses 
objectifs à sa réalisation.