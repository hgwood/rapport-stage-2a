\section{Le groupe Thales et la filiale Thales Corporate Services}

\subsection{Thales}

Thales est un leader mondial des hautes technologies pour les marchés de 
l’aéronautique et de l’espace, de la défense, de la sécurité et des transports, 
disposant d’environ 68 000 collaborateurs dans 50 pays, en France, aux 
Etats-Unis, en Amérique latine, Corée du sud et Australie. Le succès du groupe 
s'explique par sa capacité à développer des systèmes critiques multidomestiques,
un potentiel humain exceptionnel avec un haut niveau de qualification (60\% 
d’ingénieurs et cadres), des équipes multiculturelles unies par les mêmes 
valeurs et une politique de ressources humaines dynamique. Le chiffre d'affaire
du groupe est de 12.8 milliards d'euros en 2009. Les actions sont principalement
partagés entre l'état français à 27\% et Dassault Aviation à 26\%, le reste étant 
flottant.

Thales est composé de sept divisions : Systèmes de défense et sécurité, 
Opérations Aériennes, Avionique, Espace, Systèmes de mission de défense, 
Défense Terrestre, Systèmes de Transport.
Chacune de ces divisions est spécialisée dans des domaines définis par leurs 
marchés. L’entreprise dispose en plus de six directions spécialisées dans les 
fonctions transverses au groupe Thales que sont : Finance et Juridique, 
Recherche et Technologie, Opérations, Audit Interne, Stratégie, Ressources 
Humaines et Communications. Ces directions sont parfois décentralisées par pôle 
ou région du monde.

\subsection{Thales Corporate Services}

Thales Corporate Services (Thales CST ou simplement CST) est
sous la responsabilité de la direction des Opérations. C’est une filiale à 100\%
de Thales, comptant 293 employés. Il s'agit d'une entité regroupant un ensemble 
de compétences, de ressources et de moyens afin de mettre en oeuvre des services
partagés entre les entités du groupe. Dans le cadre de ses missions, Thales CST 
fournit aux entités du groupe des services appartenant à des domaines tels que :
l'"engineering" des systèmes, des logiciels et du matériel, les processus et 
méthodologies associés, le développement de solutions pour les systèmes 
d’information (infrastructure et applications) et services associés (déploiement
des solutions, pilotage des contrats de services, etc.) ainsi que les achats 
transverses. Le déploiement de ces processus et outils communs dans les 
différents pays et/ou dans les entités du groupe s’appuie sur les moyens propres
à chaque pays et/ou entités. En fonction de la stratégie de déploiement adoptée,
Thales CST interviendra afin de supporter les pays et/ou entités dans le 
déploiement de ces processus et outils communs. Les objectifs de Thales CST 
sont : la qualité des services et des produits, la tenue des engagements en 
termes de délais et de maîtrise des coûts, l’amélioration de la compétitivité du
groupe en s’appuyant sur la mutualisation des moyens.

Pour mener à bien ses missions, Thales CST est organisée en centres de 
compétences qui répondent chacun à des règles de gouvernance qui leur sont 
propres. Un centre de compétences peut fournir : 
\begin{itemize}
	\item{des services récurrents qui s’appliquent à l’ensemble des unités d’un 
	pays (achats généraux) ou à l’ensemble des unités du groupe;}
	\item{le développement de projets transverses qui intéressent plusieurs 
	unités/divisions du groupe;}
	\item{des compétences qui participeront au développement de projets dans le 
	domaine des systèmes d’information du groupe.}
\end{itemize}

Thales CST est organisée en trois centres de compétences : 

\begin{description}
\item[Purchasing] qui est placé sous l’autorité opérationnelle de la Direction 
des Achats du Groupe. Il comprend : des activités couvrant les achats indirects 
en France et liées à la mise en oeuvre de contrats et accords cadres, de 
politiques fournisseurs et les approvisionnements associés, des activités liées 
aux initiatives achats transverses au Groupe (mise en oeuvre de contrats et 
accords cadres, de politiques fournisseurs) couvrant les segments achats directs
(inclus dans les offres de Thales).
\item[Information Systems \& Process (IS\&P)] qui est placé sous l’autorité 
opérationnelle de la Direction des Systèmes d’Information du Groupe. Il 
comprend : les activités de management de la production des services 
informatiques pour le compte des unités du Groupe en France ainsi que le 
management de la production des services d’infrastructure du Groupe au niveau 
mondial, la mise à disposition de compétences en matière de conception, de 
développement et de support au déploiement des systèmes d’information et des 
infrastructures communes à plusieurs unités du Groupe.
\item[Engineering \& Process Management (EPM)] qui est chargé d’accroître la 
compétitivité des entités du groupe Thales en fournissant des solutions 
optimisées sur l’ensemble du cycle de développement des produits et systèmes. 
EPM fournit dans ce but des procédés, méthodes et des outils.
\end{description}

Mon stage s'est déroulé au sein de EPM, et plus précisément dans la département 
System \& Software, qui fournit les logiciels de développement pour les entités 
Thales à travers le monde. 

\subsection{Thales Global Services}

En août 2011, Thales Corporate Services a changé de nom et est devenu Thales Global Services. 
Ce changement fait partie d'un processus de plus grande ampleur visant à 
regrouper tous les services transverses de Thales, et en particulier la 
communication, le marketing et les ressources humaines, tout ceci au niveau 
mondial. La transformation devrait se traduire par une expansion majeure 
du centre IS\&P et par une augmentation du nombre d'employés, qui devrait 
avoisiner les 800 (contre moins de 300 actuellement) dans le courant 2012.